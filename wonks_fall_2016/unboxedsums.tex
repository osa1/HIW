\documentclass{beamer}

\usepackage[utf8]{inputenc}
\usepackage{amssymb}
\usepackage{listings}
\usepackage{minted}
\usepackage{natbib}
\usepackage{pdfpages}
\usepackage{stmaryrd}
\usepackage{tikz}

\setbeamertemplate{itemize items}[square]

\begin{document}
\title{Unboxing sum types in GHC}

\author[Ömer\,S.\,Ağacan \& Ryan\,Scott \& Ryan\,R.\,Newton]
{%
  \texorpdfstring{
    \begin{columns}
      \column{.33\linewidth}
      \centering
      Ömer S. Ağacan\\
      \href{}{}
      \column{.33\linewidth}
      \centering
      Ryan Scott\\
      \href{}{}
      \column{.33\linewidth}
      \centering
      Ryan R. Newton\\
      \href{}{}
    \end{columns}
    \begin{columns}
      \column{.50\linewidth}
      \centering
      Johan Tibell\\
      \href{}{}
      \column{.50\linewidth}
      \centering
      José Manuel Calderón Trilla \\
      \href{}{}
    \end{columns}
  }
  {Ömer\,S.\,Ağacan \& Ryan\,R.\,Newton}
}

\date{\today}

\frame{\titlepage}

%\frame{\frametitle{Table of contents}\tableofcontents}

\begin{frame}[fragile]
    \frametitle{Overview}

    \begin{itemize}
        \item
            Lazy evaluation sucks for many reasons \\
            (more on this later if I have time)
        \item
            One of the reasons is that it implies boxing \\
            (a thunk may be arbitrarily big, sizes are not necessarily the same
            even if the types are same)

        \item
            If we guarantee strictness in some cases, we can do some
            optimizations
        \item
            Previously these optimizations were only done on product types, now
            we do those on sum types
%         \item
%
%                     \begin{minted}{haskell}
% let n = fib 100
%     b = n + 1
%     c = n + 2
%  in (b, c)
%                     \end{minted}
    \end{itemize}
\end{frame}

\begin{frame}
    \frametitle{Avoiding lazy evaluation woes}

    \begin{itemize}
        \item
            GHC developers were aware of the problem, thus developed "demand
            analysis", "constructed product results analysis", "worker/wrapper
            transformations" and language extensions like "bang-patterns"
        \item
            Haskell users were aware of the problem, and they invented
            typeclasses like ``deepseq'' and wrote code like ...
    \end{itemize}
\end{frame}

\begin{frame}[fragile]
            \begin{minted}[fontsize=\footnotesize]{haskell}
-- | Use the method of Ridders to compute a root of a function.
--
-- The function must have opposite signs when evaluated at the lower
-- and upper bounds of the search (i.e. the root must be bracketed).
ridders tol (lo,hi) f
    ...
  where
    go !a !fa !b !fb !i
        = ...
      where
        !m   = a + dm
        !fm  = f m
        !dn  = signum (fb - fa) * dm * fm / sqrt (fm*fm - fa*fb)
        !n   = m - signum dn * min (abs dn) (abs dm - 0.5 * tol)
        !fn  = f n
    !flo = f lo
    !fhi = f hi
            \end{minted}

            (Real example, from ``statistics'' package)
\end{frame}

\begin{frame}[fragile]
    \frametitle{GHC 8.0 and new language pragmas}

    \begin{itemize}
        \item -XStrictData
        \item
            Before: \\
            \begin{minted}{haskell}
data D = D !Int !String
            \end{minted}

        \item
            After: \\
            \begin{minted}{haskell}
{-# LANGUAGE StrictData #-}
data D = D Int String
            \end{minted}

        \item
            Nothing interesting, just adds bang patterns.

    \end{itemize}
\end{frame}

\begin{frame}[fragile]
    \frametitle{GHC 8.0 and new language pragmas}

    \begin{itemize}
        \item -XStrict
        \item Before: \\
            \begin{minted}[fontsize=\footnotesize]{haskell}
go !a !fa !b !fb !i
    = ...
  where
    !m   = a + dm
    !fm  = f m
    !dn  = signum (fb - fa) * dm * fm / sqrt (fm*fm - fa*fb)
    !n   = m - signum dn * min (abs dn) (abs dm - 0.5 * tol)
    !fn  = f n
            \end{minted}
    \end{itemize}
\end{frame}

\begin{frame}[fragile]
    \frametitle{GHC 8.0 and new language pragmas}

    \begin{itemize}
        \item -XStrict
        \item After: \\
            \begin{minted}[fontsize=\footnotesize]{haskell}
{-# LANGUAGE Strict #-}
go a fa b fb i
    = ...
  where
    m   = a + dm
    fm  = f m
    dn  = signum (fb - fa) * dm * fm / sqrt (fm*fm - fa*fb)
    n   = m - signum dn * min (abs dn) (abs dm - 0.5 * tol)
    fn  = f n
            \end{minted}
    \end{itemize}
\end{frame}

\begin{frame}[fragile]
    \frametitle{GHC 8.0 and new language pragmas}

    \begin{itemize}
        \item -XStrict
        \item
            One exception: \\
            \begin{minted}[fontsize=\footnotesize]{haskell}
let (p, q) = ...
            \end{minted}

            Becomes \\

            \begin{minted}[fontsize=\footnotesize]{haskell}
let !(p, q) = ...
            \end{minted}
    \end{itemize}
\end{frame}

\begin{frame}
    \frametitle{Demand analysis}

    \begin{itemize}
        \item
            An analysis pass that generates argument usage information of
            functions. \\

            \[
                \begin{array}{rcll}
                    s & ::= & \perp & \textit{-- argument not used} \\
                      & |   & S & \textit{-- argument used} \\
                      & |   & S(s_1, s_2, ...) & \textit{-- argument and its fields are used} \\
                      % & |   & C(s) & \textit{-- called, result is used as specified} \\
                \end{array}
            \]
    \end{itemize}
\end{frame}

\begin{frame}[fragile]
    \frametitle{Demand analysis}
    \begin{minted}[fontsize=\footnotesize]{haskell}
data D = D Int Int

showD :: D -> String
showD (D i1 i2) = "D " ++ show i1 ++ " " ++ show i2
    \end{minted}

    Compile with \texttt{ghc -ddump-simpl -ddump-to-file -O}. \\

    \begin{minted}[fontsize=\footnotesize]{haskell}
showD [InlPrag=INLINE[0]] :: D -> String
[GblId,
 Arity=1,
 Caf=NoCafRefs,
 Str=DmdType <S(SS),1*U(1*U(U),1*U(U))>,
 ...]
showD = ...
    \end{minted}
\end{frame}

\begin{frame}[fragile]
    \frametitle{Demand analysis}
    \begin{minted}[fontsize=\footnotesize]{haskell}
data D = D Int Int

showD :: D -> String
showD D{} = "D"
    \end{minted}

    Compile with \texttt{ghc -ddump-simpl -ddump-to-file -O}. \\

    \begin{minted}[fontsize=\footnotesize]{haskell}
showD :: D -> String
[GblId,
 Arity=1,
 Str=DmdType <S,1*H>,
 ...]
showD1 = ...
    \end{minted}
\end{frame}

\begin{frame}[fragile]
    \frametitle{Demand analysis}
    \begin{minted}[fontsize=\footnotesize]{haskell}
data D = D Int Int

showD :: D -> String
showD _ = "D"
    \end{minted}

    Compile with \texttt{ghc -ddump-simpl -ddump-to-file -O}. \\

    \begin{minted}[fontsize=\footnotesize]{haskell}
showD :: D -> String
[GblId,
 Arity=1,
 Str=DmdType <L,A>,
 ...]
showD = ...
    \end{minted}
\end{frame}

\begin{frame}[fragile]
    \frametitle{Demand analysis}
    \begin{minted}[fontsize=\footnotesize]{haskell}
map :: (a -> b) -> [a] -> [b]
map _ []       = []
map f (x : xs) = f x : map f xs
    \end{minted}

    Compile with \texttt{ghc -ddump-simpl -ddump-to-file -O}. \\

    \begin{minted}[fontsize=\footnotesize]{haskell}
map :: forall a_ane b_anf. (a_ane -> b_anf) -> [a_ane] -> [b_anf]
[GblId, Arity=2, Caf=NoCafRefs,
 Str=DmdType <L,C(U)><S,1*U>]
map = ...
    \end{minted}
\end{frame}

\begin{frame}[fragile]
    \frametitle{Constructed product results analysis}

    Demand analysis was about argument passing. This analysis is about returning. \\

    \begin{minted}[fontsize=\footnotesize]{haskell}
divMod :: Int -> Int -> (Int, Int)
divMod x y = (x `div` y, x `mod` y)
    \end{minted}

Usually you use it like this: \\

    \begin{minted}[fontsize=\footnotesize]{haskell}
f =
  ...
  let (d, m) = divMod x y
  ...
    \end{minted}

    Here we say \texttt{divMod} has CPR property.

\end{frame}

\begin{frame}[fragile]
    \frametitle{Constructed product results analysis}

    (from \cite{cpr}) \\

    \begin{minted}[fontsize=\footnotesize]{haskell}
chr :: Int -> Char
chr i = if i >= 0 && i <= 255 then C# i else error "Bad arg to chr"

g :: Int -> (Int, Int)
g x = if x < 0 then (1, 1) else g (x - 1)
    \end{minted}

    Negative example: \\

    \begin{minted}[fontsize=\footnotesize]{haskell}
head :: [a] -> a
head (x : _) = x
    \end{minted}
\end{frame}

\begin{frame}
    \frametitle{Worker/wrapper transformation}

    \begin{itemize}
        \item So far we've seen two analyses but no program transformations.
        \item Worker/wrapper uses the results to generate functions with
            specialized calling conventions for efficient argument
            passing/returning.
        \item
            Worker: A function with specialized calling convention.
        \item
            Wrapper: A small function that works as ``impedance matcher'', it has
            the standard calling convention, calls the worker.
    \end{itemize}
\end{frame}

\begin{frame}[fragile]
    \frametitle{Worker/wrapper -- examples}

    \begin{minted}[fontsize=\footnotesize]{haskell}
    fac :: Int -> Int
    fac 0 = 1 -- CPR!
    fac n = n * fac (n - 1)

    Main.fac :: Int -> Int
    [GblId,
     Arity=1,
     Caf=NoCafRefs,
     Str=DmdType <S(S),1*U(1*U)>m,
     ...]
    Main.fac = ...
    \end{minted}

\end{frame}

\begin{frame}[fragile]
    \frametitle{Worker/wrapper -- examples}

    \begin{minted}[fontsize=\footnotesize]{haskell}
    fac :: Int -> Int
    fac 0 = 1 -- CPR!
    fac n = n * fac (n - 1)

    ---

    fac :: Int -> Int
    fac i =
        case i of I# i' -> I# (wfac i')

    wfac :: Int# -> Int#
    wfac i =
        case i of
          0# -> 1#
          _  -> i *# (wfac (i -# 1#))
    \end{minted}

\end{frame}

\begin{frame}[fragile]
    \frametitle{Worker/wrapper}

    It's essential that the wrapper is inlined everywhere! \\

    \begin{minted}[fontsize=\footnotesize]{haskell}
    data D = D !Int

    f = D (fac 10)
    \end{minted}

    With \texttt{-O}, \texttt{Int} is inlined. \\

    \begin{minted}[fontsize=\footnotesize]{haskell}
    f = case fac (I# 10#) of
          I# ret -> D ret
    \end{minted}

\end{frame}

\begin{frame}[fragile]
    \begin{minted}[fontsize=\footnotesize]{haskell}
    fac :: Int -> Int
    fac i = case i of I# i' -> I# (wfac i')

    f = case fac (I# 10#) of
          I# ret -> D ret

    ---> (inline fac)

    f = case (case (I# 10) of
                I# i' -> I# (wfac i')) of
          I# ret -> D ret

    ---> (case of known constructor)

    f = case (I# (wfac 10)) of
          I# ret -> D ret

    ---> (case of known constructor)

    D (wfac 10)
    \end{minted}
\end{frame}

\begin{frame}[fragile]
    \begin{minted}[fontsize=\footnotesize]{haskell}
    divMod :: Int -> Int -> (Int, Int)
    divMod x y = (x `div` y, x `mod` y)

    ---

    wdivMod :: Int -> Int -> (# Int, Int #)
    wdivMod x y = (# x `div` y, x `mod` y #)

    divMod :: Int -> Int -> (Int, Int)
    divMod x y = case wdivMod x y of
                   (# x', y' #) -> (x, y)
    \end{minted}
\end{frame}

\begin{frame}[fragile]
    \begin{minted}[fontsize=\footnotesize]{haskell}
    divMod :: Int -> Int -> (Int, Int)
    divMod x y = case wdivMod x y of
                   (# x', y' #) -> (x, y)

    let (d, m) = divMod x y
    <expr>

    --> (compile to Core)

    case divMod x y of
      (d, m) -> <expr>

    --> (inline the wrapper)

    case (case wdivMod x y of
            (# x', y' #) -> (x', y')) of
      (d, m) -> <expr>
    \end{minted}
\end{frame}

\begin{frame}[fragile]
    \begin{minted}[fontsize=\footnotesize]{haskell}
    case (case wdivMod x y of
            (# x', y' #) -> (x', y')) of
      (d, m) -> <expr>

      --> (case of case, case of known constructor, etc.)

    case wdivMod x y of
      (# d, m #) -> <expr>
    \end{minted}
\end{frame}

\begin{frame}[fragile]
    \begin{minted}[fontsize=\footnotesize]{haskell}
        let p@(d, m) = divMod x y
        ... p ... d ... m ...

        -->

        case wdivMod x y of
          (# d, m #) ->
            let p = (d, m) in
            <expr>
    \end{minted}
\end{frame}

\begin{frame}
    \frametitle{What about sums?}

    We were able to optimize products because:

    \begin{itemize}
        \item We could pass products more efficiently, by passing multiple
            arguments.

        \item We could return products more efficiently by returning unboxed
            tuples.

        \item How to do the same for sum types?
    \end{itemize}
\end{frame}

\begin{frame}
    \frametitle{Unboxed sums -- syntax}

    \[
        \begin{array}{rcll}
            sum type & ::= & \texttt{(\#} \; \tau_1 \; | \; \tau_2 \; [ \; | \; \tau_3 ... \; ] \; \texttt{\#)} \\
            sum expr & ::= & \texttt{(\#} \; e \; | \; \texttt{\#)} \texttt{\textit{-- e.g. Left case of Either}} \\
                     &  |  & \texttt{(\#} \; | \; e \; \texttt{\#)} \texttt{\textit{-- e.g. Right case of Either}} \\
                     &  |  & \texttt{(\#} \; e \; | \; | \; \texttt{\#)} \\
                     &  |  & \texttt{(\#} \; | \; e \; | \; \texttt{\#)} \\
                     &  |  & \texttt{(\#} \; | \; | \; e \; \texttt{\#)} \\
                     & ... & \\
        \end{array}
    \]
\end{frame}

\begin{frame}[fragile]
    \frametitle{Unboxed sums -- examples}

    \begin{minted}[fontsize=\footnotesize]{haskell}
f :: (# Int | Bool | (# String | [Int] #) | (# Float, Float #) #)
  -> String
f x =
  case x of
    (# i | | | #) ->
        "first alternative (i is Int)"
    (# | b | | #) ->
        "second alternative (b is Bool)"
    (# | | (# s | #) | #) ->
        "third alternative, s is String"
    (# | | (# | is #) | #) ->
        "third alternative, is is [Int]"
    (# | | | (# f1, f2 #) #) ->
        "fourth alternative, f1 and f2 are Floats"
    \end{minted}
\end{frame}

\begin{frame}[fragile]
    \frametitle{Unboxed sums -- memory layout}

    \begin{minted}[fontsize=\footnotesize]{haskell}
    (# Int | String #)
    -- tag + pointer

    (# Int# | String #)
    -- tag + Int# + pointer

    (# Int# | Float# #)
    -- tag + Int# + Float#

    (# Int | (# Bool, String #) #)
    -- tag + pointer + pointer

    (# (# Int, Float #) | (# Bool, String #) #)
    -- tag + pointer + pointer
    \end{minted}

    We can arbitrarily nest tuples and sums, no problem.
\end{frame}

\begin{frame}[fragile]
    \begin{minted}[fontsize=\footnotesize]{haskell}
    data A = A1 Int | A2 String | A3 Bool | A4 Float

    data D = D {-# UNPACK #-} !Bool
                    -- ^ now works, compiled to just a Int# (tag)
               {-# UNPACK #-} !(Either String Int)
                    -- ^ (# String | Int #)
               {-# UNPACK #-} !(A Int String Bool Float)
                    -- ^ (# Int | String | Bool | Float #)
    \end{minted}
\end{frame}

\begin{frame}[fragile]
    \frametitle{Worker/wrapper is essential}

    \begin{minted}[fontsize=\footnotesize]{haskell}
    data D = D {-# UNPACK #-} !Bool

    andD :: D -> D -> D
    andD (D b1) (D b2) = D (b1 && b2)

    --- (compile to Core)

    andD :: D -> D -> D
    andD (D b1) (D b2) =
        D (case (&& (case b1 of
                       1# -> True -- allocations!
                       2# -> False)
                    (case b2 of
                       1# -> True
                       2# -> False)) of
             True  -> 1#
             False -> 2#)
    \end{minted}
\end{frame}

\begin{frame}[fragile]
    \frametitle{Worker/wrapper is essential}

    \begin{minted}[fontsize=\footnotesize]{haskell}
    data D = D {-# UNPACK #-} !Bool

    andD :: D -> D -> D
    andD (D b1) (D b2) = D (b1 && b2)

    --- (compile to Core)

    andD :: D -> D -> D
    andD (D b1) (D b2) =
      D (wand' b1 b2) -- no packing/ unpacking, zero allocations!

    \end{minted}
\end{frame}

\begin{frame}[fragile]
    \frametitle{Evaluation}

    \begin{minted}[fontsize=\footnotesize]{haskell}
data Bx = Bx1 Int# | Bx2 Int# | Bx3 Int# | Bx4 Int# | Bx5  Int#
        | Bx6 Int# | Bx7 Int# | Bx8 Int# | Bx9 Int# | Bx10 Int#
  deriving (Show)

iter_bx :: Bx -> Int -> Bx
iter_bx bx       0 = bx
iter_bx (Bx1  x) i = iter_bx (Bx2  (x +# 1#)) (i - 1)
iter_bx (Bx2  x) i = iter_bx (Bx3  (x +# 1#)) (i - 1)
iter_bx (Bx3  x) i = iter_bx (Bx4  (x +# 1#)) (i - 1)
iter_bx (Bx4  x) i = iter_bx (Bx5  (x +# 1#)) (i - 1)
iter_bx (Bx5  x) i = iter_bx (Bx6  (x +# 1#)) (i - 1)
iter_bx (Bx6  x) i = iter_bx (Bx7  (x +# 1#)) (i - 1)
iter_bx (Bx7  x) i = iter_bx (Bx8  (x +# 1#)) (i - 1)
iter_bx (Bx8  x) i = iter_bx (Bx9  (x +# 1#)) (i - 1)
iter_bx (Bx9  x) i = iter_bx (Bx10 (x +# 1#)) (i - 1)
iter_bx (Bx10 x) i = iter_bx (Bx1  (x +# 1#)) (i - 1)

main :: IO ()
main = putStrLn (show (iter_bx (Bx1 0#) 1000000000))
    \end{minted}
\end{frame}

\begin{frame}[fragile]
    \frametitle{Evaluation}

    Total allocation goes down from 14,400,052,272 (14G) to 52,048 (52K).
    Runtime 2.44s to 1.74s.


    \begin{minted}[fontsize=\footnotesize]{text}
    _c57R:
        decq %rdi    # decrement the counter
        incq %rsi    # increment the other counter

        # GHC updates the tag in two instructions,
        # the backend is not perfect.
        movl $3,%eax
        movq %rax,%r14

        jmp _c57m    -- loop
    \end{minted}
\end{frame}

\begin{frame}[fragile]
    \frametitle{Evaluation}

    \begin{minted}[fontsize=\footnotesize]{haskell}
    safeDiv :: Int -> Int -> Maybe Int
    safeDiv _ 0 = Nothing
    safeDiv n d = Just (n `div` d)
    \end{minted}

    Unboxed variant 12\% faster.

\end{frame}

\begin{frame}
    \frametitle{Evaluation}

    \begin{itemize}
    \item
        We're getting worse results in every single program.

    \item
        Worker/wrapper is esential.

    \item
        Currently working on the extending the analyses. We believe this will
        change \texttt{\{-\# EVERYTHING \#-\}}.

    \item
        Implementation is simple (I'm doing that part) .. no idea about the
        abstract interpretation part.

    \item
        It's online! Add \texttt{\{-\# LANGUAGE UnboxedSums \#-\}}.\\
        Source: \url{https://github.com/osa1/ghc/tree/unboxed-sums-stg} \\
        Tarball (x86\_64 Linux): \url{http://cs.indiana.edu/~rrnewton/temp/ghc-8.1.20160201\_ubx\_sums\_1.2-x86\_64-unknown-linux.tar.xz}
    \end{itemize}
\end{frame}

\begin{frame}[allowframebreaks]
    \frametitle{References}

    \bibliographystyle{abbrvnat}
    \bibliography{refs}
\end{frame}

\end{document}
